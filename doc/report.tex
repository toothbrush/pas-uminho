\documentclass[a4paper]{article}


\usepackage{xspace}
\usepackage{a4wide}
\usepackage{hyperref}
\usepackage{amsmath}
\usepackage{graphicx}
\usepackage{enumerate}
\usepackage{verbatim}
\usepackage{listings}
\lstset{language=Haskell, basicstyle=\small}
\usepackage{amsfonts}
\usepackage{pifont}
\newcommand{\tickYes}{\checkmark}
\newcommand{\tickNo}{\hspace{1pt}\ding{55}}
\newcommand{\ar}{Archery\xspace}
\newcommand{\re}{Reo\xspace}
\newcommand{\mcrl}{$\mu$CRL2\xspace}

\hypersetup{
    %bookmarks=true,         % show bookmarks bar?
    unicode=false,          % non-Latin characters in Acrobat’s bookmarks
    pdftoolbar=true,        % show Acrobat’s toolbar?
    pdfmenubar=true,        % show Acrobat’s menu?
    pdffitwindow=false,     % window fit to page when opened
    pdfstartview={FitP},    % fits the page to the window?
    pdftitle={Modeling in \ar and \re},    % title
    pdfauthor={Jorge Cunha Mendes and Paul van der Walt},     % author
    pdfsubject={Model Checking},   % subject of the document
    pdfcreator={Jorge Cunha Mendes and Paul van der Walt},   % creator of the document
    pdfproducer={}, % producer of the document
    pdfkeywords=
        {\re}
        {\ar}
        {Modeling}, % list of keywords
    pdfnewwindow=true, % links in new window
    colorlinks=true,   % false: boxed links; true: colored links
    linkcolor=blue,    % color of internal links
    citecolor=blue,    % color of links to bibliography
    filecolor=blue,    % color of file links
    urlcolor=blue      % color of external links
}

\author{Jorge Mendes \and Paul van der Walt}
\date{\today}
\title{Modeling in \ar and \re}
\begin{document}

\maketitle

%
% Intro
%
\section{Introduction}

This project aims to model a system using two distinct frameworks, namely \ar
and \re, and to compare both approaches.

\ar is a modeling language developed at the University of Minho, which is designed
to be a higher-level modeling language for software architectures. There is a translator
that converts \ar models to \mcrl, making it possible to use a set of tools that work with
\mcrl to check properties about the developed models and also to simulate and visualize them.

\re is a research activity of the SEN3 research group at the CWI, in the
Netherlands. It provides a paradigm for composition of distributed software
components and services based on the notion of mobile channels.
From this research activity, resulted the \re coordination language and the
Extensible Coordination Tools (ECT) which are a set of plug-ins for the Eclipse
platform to deal with the coordination language.

The chosen system is a dynamically adjustable cluster of servers for a news
agency. Depending on the load of the system, it should add or remove servers
from the cluster. If the load is too high with all the servers available for
the cluster, then the servers should respond without media content, i.e., they
should respond with only the text of the requested page (media mode vs. text
mode).



%
% Approach
%
\section{Approach}

There are several possible approaches to model the news system. The ones that we
thought of are all based on the same idea: clients, servers and a load balancer. The
functionalities and arrangement of these elements for each architecture is what
changes between the approaches.

In the first approach, we can make the clients and the servers communicate with
each other
directly. Also, the servers communicate with the load balancer to
know if they should accept more requests and which mode (full-media or
text-only) should be used to provide the response. The architecture of this
approach is depicted in Figure \ref{fig:cslb}. However, this approach needs
a lot of connections from the clients to servers.

\begin{figure}[htb]
	\begin{center}
		\includegraphics[width=0.5\textwidth]{images/c_s_lb.png}
	\end{center}
	\caption{Client $\leftrightarrow$ Server $\leftrightarrow$ Load balancer }
	\label{fig:cslb}
\end{figure}

The second approach, depicted in Figure \ref{fig:clbs}, is the one that we
tried to implement, and provides a single point of communication between the
clients and the servers. It results in less channels for communication, but the
load balancer needs to be aware of the clients identification to route the
requests/responses correctly.

\begin{figure}[htb]
	\begin{center}
		\includegraphics[width=0.5\textwidth]{images/c_lb_s.png}
	\end{center}
	\caption{Client $\leftrightarrow$ Load balancer  $\leftrightarrow$ Server }
	\label{fig:clbs}
\end{figure}


\subsection{\ar Approach}

The basic approaches described above were modeled in \ar. However, due to some
limitations in \ar (or the translator), we've not been able to create a full
model of the system.

We first tried the approach described in Figure \ref{fig:cslb}, but we hadn't
enough knowledge at the time about out to implement the identification of the
clients.

Then we tried the other approach (Figure \ref{fig:clbs}), which we dropped because we had to set all
the attachments by hand and that version has too many connections.

After those failed attempts, we tried again the approach from the first
attempt, but we couldn't complete it because of the language / translator
limitations.

The final model uses client and server elements, which are basically the same
thing. The server waits for a request and responds, a client sends a request and
waits for a response. This is, in the limit case, the same behaviour. The
servers and clients communicate via the load balancer, which is composed of a
buffer and a balancer. The buffer just maintains a worklist, and the balancer
non-deterministically distributes a job to a server, relays a response back to
the client from the server, or receives a new request. Here we can differentiate
between requests, since they are modeled as integers. 


\subsection{\re Approach}

We tried a number of approaches in \re, and had varying degrees of success. The
current incarnation, and so far most successful, is using the following
architecture. In Figure \ref{fig:reo} one can see the global layout of the \re
model. We have the clients on one side, which connect to the load balancer.
There is an 'internet' entity in between which models that requests arrive at
the load balancer sequentially. The load balancer then distributes the requests
to one of the servers in the pool. Unfortunately, some of the things we would
have liked to model weren't possible using \re. For example, we have two
versions of the load balancer, one which nondeterministically distributes the
request to one of the servers which is in a suitable state to receive a request,
and the other version which distributes the request in a round-robin fashion.
Both have disadvantages, since in the first case we're actually modeling a
'magic' load balancer, and provide no details on how this should be implemented
in real life. One way could be for the load balancer to maintain a work list of
unanswered requests, which servers could poll as soon as they have time. This is
unrealistic, however, since in real life the servers usually don't have control
over when requests arrive. The second method (round robin) has a pitfall which
is more related to our lack of experience with \re. The problem is that if it is a
given servers turn, but that server is still busy servicing the last request,
the load balancer blocks until the server is free again. This is undesirable,
since another server might be idle. It would, however, be easier to implement
this solution in real life.

In the actual \re model, servers are represented by two components. The one
component just receives the request and passes it along to the actual server
process, but will not accept another request until the server has delivered some
response. The servers are modeled as entities with 2 FIFOs, one representing
the media mode, and the other representing text mode. In \re this makes no
difference, but if one exports the model to \mcrl, probably a different delay
per channel could be
implemented there.

The other thing that was needed, was to somehow send the server's response back
to the right client. The way this was solved in the \re model is by creating a
connector which receives all the responses and filters them. Unfortunately \re
doesn't actually know what value a given token has, even though this is
reflected in the generated \mcrl code. The result is that the filter is seen as
a pipe which may, or may not, transmit each packet, so an alternative animation
is generated each time. This is also not the way things usually work in real
life, but to keep the load balancer model simple, we chose this solution.

Also note the 'request sequencer', which just serves to collect the incoming
requests and feed them to the load balancer one at a time. This is accomplished
by a FIFO$_k$ buffer, where $k$ should be large enough to hold any number of
requests which might be incoming. In the model, only 3 buffers are used to try
and limit the state space. Also, only 3 clients and 3 servers are used, but even
this model couldn't be animated in reasonable time without simplifying it quite
drastically (removing servers and clients).

In this model, clients are modeled as a reader/writer pair, with the reader
having as many requests as the writer. In \re, this isn't sufficient to make
sure the \emph{correct} $n$ requests are received, but the framework for
checking this is present (the filters), and this could be patched up later in
\mcrl. We also don't explicitly support adding and removing servers from the
pool, but we consider an idle server as being able to turn itself off.

\begin{figure}[h]
    \begin{center}
        \includegraphics[width=0.8\textwidth]{images/reo-model.pdf}
    \end{center}
    \caption{Summary of the \re model.}
    \label{fig:reo}
\end{figure}

\subsection{\mcrl Approach}
For comparison purposes, we also modeled the problem in \mcrl to check how
different the approaches would be compared to a previously learned modeling language.

The \mcrl model is easier to read, in our opinion, and has less syntactic
sugar, needing just a little bit more code to provide some features
available in \ar.

It is composed of a network, which has a given number of clients running in
parallel, a load balancer and several servers also running in parallel. These
servers have an inner process to be able to receive requests and do long
processing at the same time.


%
% Archery and Reo
%
\section{Evaluating \ar and \re}


After having tried both tool sets extensively, we can provide a comparison between
the pros and cons of each. Both tools have a glaring lack of documentation.
Notably \re, which seems to be a relatively serious and usable tool, only has a
few very basic examples scattered around, and no information regarding the use
of
programmable components in larger models. Both frameworks also seem to lack an
import feature, which would aid in making each part of the model more readable.

The first small and obvious frustration with \ar is its lack of support for
comments in the model. Also, it's unfortunate that it is impossible to
dynamically create and remove connections, which would make a more direct
translation of our idea of dynamic servers possible. There are also a number of
bugs/limitations which make using the tool a challenge. For example, the error
reporting is rather fragile, so often one encounters a null pointer exception when there was a
parsing error. Also, one can only parameterise processes, not equations nor
other \mcrl elements. Problems may also arise, since \textit{eqn} or
\textit{map} definitions which are meant to be local to an element are placed at
the top level of the \mcrl file, which makes them global and therefore they
might interfere. It is also impossible to define constants outside of patterns
(for example inside architectures).

% Reo
\subsection{\re}


% Archery vs. Reo
\subsection{\ar vs. \re}

\ar winning\dots

%
% Conclusion
%
\section{Conclusion}
We couldn't provide a complete model in each framework, but we were able to
experiment with them and find some of their limits. However, we did not use all the
functionalities from each framework due to lack of documentation. For \re there
is some documentation, but only about the coordination language, and not about
the tools to work with the language.

Both \ar and \re provide a translation to \mcrl. \re provides correct \mcrl
code whilst \ar may be translated to invalid \mcrl, probably because there is
no type check.

\ar provides a better readability of the models code, with better organization
of it, comparing to \mcrl models. However, \ar misses some features available
in \mcrl, notably summands and parallelization, and the language or translator is
fairly limited. Also, a graphical interface would improve the development of
\ar models a lot (for example, all the connections can become quite a mess in
\ar).


\re is powerful, due to the possibility of programming the components in Java,
but we didn't find a way to use those components in our model using the ECT nor
using ReoLive. Another limitation that we found in \re is the fact that it
doesn't have proper support for values like integers in the tokens using the
ECT, or at least the animation ignores them. They are however passed through to
\mcrl, so one can write properties about them. The graphical editor available in
the ECT provides an easy to read visualization of the model, but lacks in some
basic functionalities like copy/paste. We also were unable to create Components
out of our Connectors (in some cases it works, in others it doesn't), and even
when it does work, we couldn't create a basic building block in one \re file and
use it in another.  This is probably due to a lack of documentation, because
it's likely to be possible somehow.



\appendix
\section{Full \re model}\label{app:reo}


    \begin{center}
        \includegraphics[height=0.95\textheight]{images/reo-full.png}
    \end{center}
\end{document}
