\documentclass[a4paper]{article}


\usepackage{xspace}
\usepackage{a4wide}
\usepackage{hyperref}
\usepackage{amsmath}
\usepackage{graphicx}
\usepackage{enumerate}
\usepackage{verbatim}
\usepackage{listings}
\lstset{language=Haskell, basicstyle=\small}
\usepackage{amsfonts}
\usepackage{pifont}
\newcommand{\tickYes}{\checkmark}
\newcommand{\tickNo}{\hspace{1pt}\ding{55}}
\newcommand{\ar}{Archery\xspace}
\newcommand{\re}{Reo\xspace}
\newcommand{\mcrl}{$\mu$CRL2\xspace}

\hypersetup{
    %bookmarks=true,         % show bookmarks bar?
    unicode=false,          % non-Latin characters in Acrobat’s bookmarks
    pdftoolbar=true,        % show Acrobat’s toolbar?
    pdfmenubar=true,        % show Acrobat’s menu?
    pdffitwindow=false,     % window fit to page when opened
    pdfstartview={FitP},    % fits the page to the window?
    pdftitle={Modeling in \ar and \re},    % title
    pdfauthor={Jorge Cunha Mendes and Paul van der Walt},     % author
    pdfsubject={Model Checking},   % subject of the document
    pdfcreator={Jorge Cunha Mendes and Paul van der Walt},   % creator of the document
    pdfproducer={}, % producer of the document
    pdfkeywords=
        {\re}
        {\ar}
        {Modeling}, % list of keywords
    pdfnewwindow=true, % links in new window
    colorlinks=true,   % false: boxed links; true: colored links
    linkcolor=blue,    % color of internal links
    citecolor=blue,    % color of links to bibliography
    filecolor=blue,    % color of file links
    urlcolor=blue      % color of external links
}

\author{Jorge Mendes \and Paul van der Walt}
\date{\today}
\title{Modeling in \ar and \re}
\begin{document}

\maketitle

%
% Intro
%
\section{Introduction}

This project aims to model a system using two distinct frameworks, namely \ar
and \re, and to compare both approaches.

\ar is a modeling language developed at the University of Minho, which is designed
to be a higher-level modeling language for software architectures. There is a translator
that converts \ar models to \mcrl, making it possible to use a set of tools that work with
\mcrl to check properties about the developed models and also to simulate and visualize them.

\re is a research activity of the SEN3 research group at the CWI, in the
Netherlands. It provides a paradigm for composition of distributed software
components and services based on the notion of mobile channels.
From this research activity, resulted the \re coordination language and the
Extensible Coordination Tools (ECT) which are a set of plug-ins for the Eclipse
platform to deal with the coordination language.

The chosen system is a dynamically adjustable cluster of servers for a news
agency. Depending on the load of the system, it should add or remove servers
from the cluster. If the load is too high with all the servers available for
the cluster, then the servers should respond without media content, i.e., they
should respond with only the text of the requested page (media mode vs. text
mode).

%
% The problem
%
%\section{The Problem}
%
%The problem was to model a news website, which consists of users, a load
%balancer, and a number of servers. There is some maximum $M$ of servers, and
%when the load balancer detects that the network is busy, servers can dynamically
%be added to the pool up to this maximum $M$. A server can also answer in media
%or text mode, depending on the load conditions. The aim is to guarantee that all
%requests will be serviced within a certain time limit $x$, by adding servers
%when necessary, spreading the load evenly, and switching down to text-mode when
%the servers are too busy. A server can respond to a given number of requests per
%minute in media mode, and $asdf$ per minute in text mode (with $asdf<asdf$

% XYZ News is proud to be the reliable online news agency to first provide information in different formats to its audience. Their journalists arrive where news take place and prepare text, images and videos to be put online. However, their reputation has a price. Whenever an important event takes place, the website receives massive amounts of requests for a short period of time. XYZ applies two tactics to manage these peaks. The first is to dinamically add up to N web servers to the cluster and the second is to provide only text content in articles. Then, a server can be attending requests in either a full-media or a text-only mode. The first tactic is used until there are no more available servers. Then, the second is applied. XYZ board wants that every request is answered, and if possible, in less than n (n > m) miliseconds. They know that a server can respond in less than m miliseconds up to R simultaneous requests in full-media mode, and up to Q in the text-only one.

%
% Approach
%
\section{Approach}

% TODO describe generic approach
There can be several approaches to model the news system. The ones that we
though of are based on the same idea: client, servers and a load balancer. The
functionalities and arrangement of this elements for each architecture is what
change between the approaches.

In the first approach, we can make the clients and the servers communicate
directly between them. Also, the servers communicate with the load balancer to
know if it should accept more requests and wich kind of mode (full-media or
text-only) should be used to provide the response. The architecture of this
approach is depicted in figure \ref{fig:cslb}. However, this approach needs
a lot of connections from the clients to servers.

\begin{figure}[htb]
	\begin{center}
		\includegraphics[width=0.5\textwidth]{images/c_s_lb.png}
	\end{center}
	\caption{C S LB}
	\label{fig:cslb}
\end{figure}

The second approach, depicted in figure \ref{fig:clbs}, is the one that we
tried to implement, and provides an unique point of communication between the
clients and the servers. It results in less channels for communication, but the
load balancer needs to be aware of the clients identification to route the
requests/reponses correctly.

\begin{figure}[htb]
	\begin{center}
		\includegraphics[width=0.5\textwidth]{images/c_lb_s.png}
	\end{center}
	\caption{C LB S}
	\label{fig:clbs}
\end{figure}

% TODO describe Archery approach(es)

% TODO describe Reo approach(es)

%
% Archery and Reo
%
\section{Evaluating \ar and \re}

In this section we will detail the approaches we used to model the problem in
both \ar and \re, as well as design decisions and simplifications we made to the
original problem.



% Archery
\subsection{\ar}


% Reo
\subsection{\re}


% Archery vs. Reo
\subsection{\ar vs. \re}

\ar winning\dots

%
% Conclusion
%
\section{Conclusion}
\ar provides a better readability of the models code, with better organization of it,
comparing to \mcrl models.

However, \ar misses some features available in \mcrl, namely summands and paralelization.

A graphical interface would improve a lot the development of Archery models.

Make a note about the fact that \re cannot handle values in the tokens! %todo

\end{document}
