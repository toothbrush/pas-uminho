\documentclass[a4paper]{article}


\usepackage{xspace}
\usepackage{a4wide}
\usepackage{hyperref}
\usepackage{amsmath}
\usepackage{graphicx}
\usepackage{enumerate}
\usepackage{verbatim}
\usepackage{listings}
\lstset{language=Haskell, basicstyle=\small}
\usepackage{amsfonts}
\usepackage{pifont}
\newcommand{\tickYes}{\checkmark}
\newcommand{\tickNo}{\hspace{1pt}\ding{55}}
\newcommand{\ar}{Archery\xspace}
\newcommand{\re}{Reo\xspace}
\newcommand{\mcrl}{mCRL2\xspace}



\hypersetup{
    %bookmarks=true,         % show bookmarks bar?
    unicode=false,          % non-Latin characters in Acrobat’s bookmarks
    pdftoolbar=true,        % show Acrobat’s toolbar?
    pdfmenubar=true,        % show Acrobat’s menu?
    pdffitwindow=false,     % window fit to page when opened
    pdfstartview={FitP},    % fits the page to the window?
    pdftitle={Modeling in \ar and \re},    % title
    pdfauthor={Jorge Cunha Mendes and Paul van der Walt},     % author
    pdfsubject={Model Checking},   % subject of the document
    pdfcreator={Jorge Cunha Mendes and Paul van der Walt},   % creator of the document
    pdfproducer={}, % producer of the document
    pdfkeywords=
        {\re}
        {\ar}
        {Modeling}, % list of keywords
    pdfnewwindow=true, % links in new window
    colorlinks=true,   % false: boxed links; true: colored links
    linkcolor=blue,    % color of internal links
    citecolor=blue,    % color of links to bibliography
    filecolor=blue,    % color of file links
    urlcolor=blue      % color of external links
}


\author{Jorge Mendes \and Paul van der Walt}
\date{\today}
\title{Modeling in \ar and \re}
\begin{document}

\maketitle

\section{Introduction}

This project aims to implement a systems model in two different modeling tools,
namely \ar and \re.

\subsection{\ar}

\ar is a modeling language developed at the University of Minho, which is designed
to be a higher-level modeling language for software architectures. There is a translator
that converts \ar models to \mcrl, making possible to use a set of tools that work with
\mcrl to check properties about the developed models and also to simulate and visualize them.

\subsection{\re}

\re is a research activity of the SEN3 research group at the CWI, in the
Netherlands. It provides a paradigm for composition of distributed software
components and services based on the notion of mobile channels.

From this research activity, resulted the \re coordination language and the
Extensible Coordination Tools (ECT) which are a ser of plug-ins for the Eclipse
platform to deal with the coordination language.

\section{The problem}

The problem was to model a news website, which consists of users, a load
balancer, and a number of servers. There is some maximum $M$ of servers, and
when the load balancer detects that the network is busy, servers can dynamically
be added to the pool up to this maximum $M$. A server can also answer in media
or text mode, depending on the load conditions. The aim is to guarantee that all
requests will be serviced within a certain time limit $x$, by adding servers
when necessary, spreading the load evenly, and switching down to text-mode when
the servers are too busy. A server can respond to a given number of requests per
minute in media mode, and $asdf$ per minute in text mode (with $asdf<asdf$

\section{Conclusion}

\re will win. 

\end{document}
